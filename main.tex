\documentclass{report}


\usepackage[italian]{babel}

% Set page size and margins
% Replace `letterpaper' with `a4paper' for UK/EU standard size
\usepackage[letterpaper,top=2cm,bottom=2cm,left=3cm,right=3cm,marginparwidth=1.75cm]{geometry}

% Useful packages
\usepackage{amsmath}
\usepackage{graphicx}
\usepackage[colorlinks=true, allcolors=blue]{hyperref}
\usepackage[T1]{fontenc}
\usepackage[utf8]{inputenc}
\usepackage{setspace}
\usepackage[paper=a4paper,margin=1in]{geometry}
\usepackage[square,numbers,super]{natbib}
\usepackage[nottoc]{tocbibind}



\begin{document}


\begin{titlepage}
        
        \noindent
        \begin{minipage}[t]{0.19\textwidth}
            \vspace{-4mm}{\includegraphics[scale=1.15]{logo_unimib.pdf}}
        \end{minipage}
        \begin{minipage}[t]{0.81\textwidth}
        {
                \setstretch{1.42}
                {\textsc{Università degli Studi di Milano - Bicocca}} \\
                \textbf{Scuola di Scienze} \\
                \textbf{Dipartimento di Informatica, Sistemistica e Comunicazione} \\
                \textbf{Corso di laurea in Informatica} \\
                \par
        }
        \end{minipage}
        
	\vspace{40mm}
        
	\begin{center}
            {\LARGE{
                    \setstretch{1.2}
                    \textbf{Strategie di ottimizzazione ed automatizzazione di liquidità nella finanza decentralizzata}
                    \par
            }}
        \end{center}
        
        \vspace{50mm}

        \noindent
        {\large \textbf{Relatore:} Prof. Alberto Leporati } \\

        \noindent
        {\large \textbf{Correlatore:} Per ora nessuno}
        
        \vspace{15mm}

        \begin{flushright}
            {\large \textbf{Relazione della prova finale di:}} \\
            \large{Christian Kobril} \\
            \large{Matricola 856448} 
        \end{flushright}
        
        \vspace{40mm}
        \begin{center}
            {\large{\bf Anno Accademico 2021-2022}}
        \end{center}

        \restoregeometry
        
    \end{titlepage}

\tableofcontents

\begin{enumerate}
\item Introduzione alla Finanza Decentralizzata
\item AMM e pool di liquidità
\item Complessità di Uniswap v3
\item Orbit, piattaforma per automatizzare e ottimizzare strategie defi
\item Utilizzo degli smart vaults
\item Moduli di Orbit: autocompound, rebalance and idle liquidity
\item Tecnologie utilizzate nello sviluppo di Orbit
\item Architettura degli smart contracts
\item Future implementazioni all'interno di Orbit
\end{enumerate}





\chapter{Introduzione}
Il tema centrale di questa tesi di laurea è la \textbf{finanza decentralizzata}.\\ In particolare, si approfondirà il ruolo che la blockchain ha in questo settore, fornendo esempi di un prodotto software concretamente sviluppato durante il mio  \textit{Project Work} svolto presso l'azienda \textit{Five Elements Labs Srl}.
\\\\Inoltre, verranno approfonditi i concetti di \textit{pool di liquidità}, \textit{automated market maker(AMM)} e \textit{strategie DeFi} su Uniswap v3.

\section{Cos'è la Finanza Decentralizzata}

Il termine "finanza decentralizzata" viene usato per classificare tutti quei servizi finanziari che avvengono direttamente tra due enti sulla blockchain.

\subsection{Differenze tra Finanza Decentralizzata e Tradizionale}

Per comprendere il concetto su cui si basa la finanza decentralizzata (da ora \textit{DeFi}), è bene dare un rapido sguardo alla sua controparte: la finanza tradizionale (o centralizzata).\\\\
Nella finanza centralizzata, ogni singola operazione finanziaria tra due persone (bonifici, prestiti, scambio di risorse, mutui) richiede l'interazione con soggetti di terze parti (tipicamente banche o altri enti).\\
Ciò incrementa le già prolisse tempistiche burocratiche, oltre ad aggiungere i costi dovuti al servizio fornito dagli enti che permettono l'operazione.
\\\\D'altra parte, la finanza decentralizzata permette l'interazione tra due soggetti senza l'intermediazione di un sistema centralizzato, bensì mediante un applicativo software costruito sopra la tecnologia blockchain, rendendo le operazioni rapide, pubbliche e sicure.


\section{Caratteristiche e vantaggi della DeFi}

\subsection{Applicazioni decentralizzate}
Gli applicativi software utilizzati in DeFi vengono definitivi \textit{dApps (decentralized applications)}, ovvero particolari prodotti che utilizzano la blockchain di Ethereum\cite{ethereum}, una delle principali criptovalute, nota per la sua flessibilità e accessibilità.\\\\
Su tale blockchain risiedono dei particolari programmi denominati \textit{Smart Contracts}\cite{smartcontracts}, i quali si occupano di garantire sicurezza, trasparenza e irreversibilità delle operazioni avvenute sulla blockchain Ethereum.\\

\subsection{Accesso alle dApps}

Il concetto di login ideato nel web2, tipicamente caratterizzato dall'inserimento di un'email e una password, viene sorpassato da una nuova autenticazione del web3\cite{web2_web3}: attraverso il proprio \textit{portafoglio digitale} (da ora, wallet \cite{wallet}) è possibile accedere al proprio account e gestire i propri assets digitali, eventualmente mettendoli a disposizione della dApp a cui si è connessi.
\\\\I wallet presentano il vantaggio di non dover fornire nomi, indirizzi fisici, email o altre informazioni personali, garantendo la riservatezza dei propri dati; basta creare un wallet per avere immediato accesso alle piattaforme, senza registrazioni.



\subsection{Operazioni in DeFi}

Ogni operazione viene detta \textit{"transazione"}. Una transazione è permanentemente salvata sui registri della blockchain, rendendo ogni singola transazione, associata ad un identificativo, consultabile in qualsiasi momento da qualsiasi persona.
\\Tale trasparenza viene difficilmente concessa dalle banche, ponendo la DeFi come un sistema aperto e rintracciabile.

\subsection{Flessibilità}

L'ultima caratteristica della DeFi che ritengo importante citare è ciò che più la contraddistingue dalla finanza tradizionale: la sua flessibilità.\\
In qualsiasi momento un utente può trasferire i propri assets digitali, senza dover chiedere il permesso a soggetti di terze parti, evitando costose commissioni e con un attesa che va dai pochi secondi ai pochi minuti. 


\chapter{Servizi forniti nel settore DeFi}

Avendo ora in mente cosa è la DeFi, quali sono le sue caratteristiche e i principali vantaggi, è bene approfondire quali sono i prodotti "dApps" che hanno influenzato il lavoro svolto durante il mio Project Work.

\section{Uniswap}

Innanzitutto, onde evitare facili fraintendimenti, è bene distinguere la piattaforma Uniswap (\hyperlink{Uniswap}{https://app.uniswap.org/}), ossia ciò con cui l'utente interagisce direttamente, dall'omonimo protocollo.

\subsection{Protocollo Uniswap}



\chapter{Orbit, piattaforma per automatizzare le strategie DeFi}

\chapter{Tecnologie utilizzate nello sviluppo di Orbit}

\chapter{Architettura degli Smart Contracts di Orbit}

\chapter{Future implementazione all'interno di Orbit}


\bibliographystyle{unsrtnat}
\bibliography{bibliografia}




\end{document}