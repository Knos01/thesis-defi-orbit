\documentclass{report}


\usepackage[italian]{babel}

% Set page size and margins
% Replace `letterpaper' with `a4paper' for UK/EU standard size
\usepackage[letterpaper,top=2cm,bottom=2cm,left=3cm,right=3cm,marginparwidth=1.75cm]{geometry}

% Useful packages
\usepackage{amsmath}
\usepackage{graphicx}
\usepackage[colorlinks=true, allcolors=blue]{hyperref}
\usepackage[T1]{fontenc}
\usepackage[utf8]{inputenc}
\usepackage{setspace}
\usepackage[paper=a4paper,margin=1in]{geometry}
\usepackage[square,numbers,super]{natbib}
\usepackage[nottoc]{tocbibind}



\begin{document}


\begin{titlepage}
        
        \noindent
        \begin{minipage}[t]{0.19\textwidth}
            \vspace{-4mm}{\includegraphics[scale=1.15]{logo_unimib.pdf}}
        \end{minipage}
        \begin{minipage}[t]{0.81\textwidth}
        {
                \setstretch{1.42}
                {\textsc{Università degli Studi di Milano - Bicocca}} \\
                \textbf{Scuola di Scienze} \\
                \textbf{Dipartimento di Informatica, Sistemistica e Comunicazione} \\
                \textbf{Corso di laurea in Informatica} \\
                \par
        }
        \end{minipage}
        
	\vspace{40mm}
        
	\begin{center}
            {\LARGE{
                    \setstretch{1.2}
                    \textbf{Strategie di automatizzazione di liquidità nella finanza decentralizzata}
                    \par
            }}
        \end{center}
        
        \vspace{50mm}

        \noindent
        {\large \textbf{Relatore:} Prof. Alberto Leporati } \\

        \noindent
        {\large \textbf{Correlatore:} Per ora nessuno}
        
        \vspace{15mm}

        \begin{flushright}
            {\large \textbf{Relazione della prova finale di:}} \\
            \large{Christian Kobril} \\
            \large{Matricola 856448} 
        \end{flushright}
        
        \vspace{40mm}
        \begin{center}
            {\large{\bf Anno Accademico 2021-2022}}
        \end{center}

        \restoregeometry
        
    \end{titlepage}

\tableofcontents

\begin{enumerate}
\item Introduzione alla Finanza Decentralizzata
\item AMM e pool di liquidità
\item Complessità di Uniswap v3
\item Orbit, piattaforma per automatizzare e ottimizzare strategie defi
\item Utilizzo degli smart vaults
\item Moduli di Orbit: autocompound, rebalance and idle liquidity
\item Tecnologie utilizzate nello sviluppo di Orbit
\item Architettura degli smart contracts
\item Future implementazioni all'interno di Orbit
\end{enumerate}





\chapter{Introduzione}
Il tema centrale di questa tesi di laurea è la \textbf{finanza decentralizzata}.\\ In particolare, si approfondirà il ruolo che la blockchain ha in questo settore, fornendo esempi di un prodotto software concretamente sviluppato durante il mio  \textit{Project Work} svolto presso l'azienda \textit{Five Elements Labs Srl}.
\\\\Inoltre, verranno approfonditi i concetti di \textit{pool di liquidità}, \textit{automated market maker(AMM)} e \textit{strategie DeFi} su Uniswap v3.

\section{Cos'è la Finanza Decentralizzata}

Il termine "finanza decentralizzata" viene usato per classificare tutti quei servizi finanziari che avvengono direttamente tra due enti sulla blockchain.

\subsection{Differenze tra Finanza Decentralizzata e Tradizionale}

Per comprendere il concetto su cui si basa la finanza decentralizzata (da ora \textit{DeFi}), è bene dare un rapido sguardo alla sua controparte: la finanza tradizionale (o centralizzata).\\\\
Nella finanza centralizzata, ogni singola operazione finanziaria tra due persone (bonifici, prestiti, scambio di risorse, mutui) richiede l'interazione con soggetti di terze parti (tipicamente banche o altri enti).\\
Ciò incrementa le già prolisse tempistiche burocratiche, oltre ad aggiungere i costi dovuti al servizio fornito dagli enti che permettono l'operazione.
\\\\D'altra parte, la finanza decentralizzata permette l'interazione tra due soggetti senza l'intermediazione di un sistema centralizzato, bensì mediante un applicativo software costruito sopra la tecnologia blockchain, rendendo le operazioni rapide, pubbliche e sicure.


\section{Caratteristiche e vantaggi della DeFi}

\subsection{Applicazioni decentralizzate}
Gli applicativi software utilizzati in DeFi vengono definitivi \textit{dApps (decentralized applications)}, ovvero particolari prodotti che utilizzano la blockchain di Ethereum\cite{ethereum}, una delle principali criptovalute, nota per la sua flessibilità e accessibilità.\\\\
Su tale blockchain risiedono dei particolari programmi denominati \textit{Smart Contracts}\cite{smartcontracts}, i quali si occupano di garantire sicurezza, trasparenza e irreversibilità delle operazioni avvenute sulla blockchain Ethereum.\\

\subsection{Accesso alle dApps}

Il concetto di login ideato nel web2, tipicamente caratterizzato dall'inserimento di un'email e una password, viene sorpassato da una nuova autenticazione del web3\cite{web2_web3}: attraverso il proprio \textit{portafoglio digitale} (da ora, wallet \cite{wallet}) è possibile accedere al proprio account e gestire i propri assets digitali, eventualmente mettendoli a disposizione della dApp a cui si è connessi.
\\\\I wallet presentano il vantaggio di non dover fornire nomi, indirizzi fisici, email o altre informazioni personali, garantendo la riservatezza dei propri dati; basta creare un wallet per avere immediato accesso alle piattaforme, senza registrazioni.



\subsection{Operazioni in DeFi}

Ogni operazione viene detta \textit{"transazione"}. Una transazione è permanentemente salvata sui registri della blockchain, rendendo ogni singola transazione, associata ad un identificativo, consultabile in qualsiasi momento da qualsiasi persona.
\\Tale trasparenza viene difficilmente concessa dalle banche, ponendo la DeFi come un sistema aperto e rintracciabile.

\subsection{Flessibilità}

L'ultima caratteristica della DeFi che ritengo importante citare è ciò che più la contraddistingue dalla finanza tradizionale: la sua flessibilità.\\
In qualsiasi momento un utente può trasferire i propri assets digitali, senza dover chiedere il permesso a soggetti di terze parti, evitando costose commissioni e con un attesa che va dai pochi secondi ai pochi minuti. 


\chapter{Uniswap, piattaforma di Liquidity Providing}

Avendo approfondito cosa è la DeFi, quali sono le sue caratteristiche e i principali vantaggi, è bene concentrarsi su quali sono i prodotti "dApps" che hanno messo le basi per il lavoro svolto durante il mio Project Work.

\section{Uniswap}

Innanzitutto, è bene distinguere la piattaforma Uniswap\cite{uniswap} dall'omonimo protocollo.
\\La dApp di Uniswap (conosciuta come Uniswap Interface) è una piattaforma che permette agli utenti l'interazione con il protocollo Uniswap.
\\\\Quest'ultimo è bensì una suite di Smart Contracts, per definizione persistenti e non aggiornabili, che insieme costruiscono un Automated Market Maker (da ora AMM)\cite{amm}.

\section{Scambi nei mercati tradizionali}

La maggior parte dei mercati tradizionali ad accesso pubblico utilizza ciò che viene definito \textit{Order Book}\cite{order_book}, ossia un elenco degli ordini di acquisto e vendita attualmente aperti per un asset, organizzati per prezzo.
\\Sostanzialmente, un \textit{sistema di corrispondenza}\cite{matching} si occupa di abbinare gli ordini di acquisto con quelli di vendita, usando l'order book per eseguire le operazioni tra i partecipanti dello scambio.


\subsection{Automated Market Maker}

La rivoluzione introdotta dalla blockchain sta nella possibilità di creare nuovi tipi di scambi che abbinano algoritmicamente ordini di acquisto e vendita utilizzando gli smart contracts.
\\Tali scambi vengono detti \textit{Scambi Decentralizzati (DEX)}.
\\\\Un AMM è un protocollo DEX che si basa su un algoritmo di valutazione per prezzare gli asset mediante una formula matematica.
\\Il citato order book viene rimpiazzato con una pool di liquidità\cite{liquidity_pool}, contenente due asset, entrambi valuti l'uno rispetto all'altro.
\\Quando un asset viene scambiato per un altro, i prezzi relativi dei due asset cambiano, e viene determinato un nuovo tasso di mercato per entrambi.
In questo modo acquirenti e venditori interagiscono direttamente con la pool (e di conseguenza gli smart contracts), senza dover interagire tra di loro.

\subsection{Esempio pratico di AMM}

Un esempio pratico che mi ha aiutato a comprendere il funzionamento degli AMM è quello dei contadini di mele e patate.
Immaginiamo di essere un contadino e di coltivare solo patate, senza la possibilità di mangiare nient'altro.\\ Un giorno ci viene proposto di effettuare degli scambi con un venditore di mele attraverso un messaggero, il quale decide di custodire mele e patate in un contenitore magico, in modo tale che rimangano a disposizione senza marcire (e dunque perdere di valore).
\\\\La regola fondamentale per questo scambio è che il contenitore dovrà sempre contenere la stessa quantità di mele e patate.
Tale regola è in realtà la formula alla base dell'AMM di Uniswap, conosciuta come \textbf{Constant Product Formula}: 

\[ x * y = k \]
dove x corrisponde al numero di mele e y al numero di patate nel contenitore.
\\Inizialmente il contenitore sarà perfettamente bilanciato, per esempio con 500 mele e 500 patate, entrambe prezzate ad €1 per un valore totale di €250.000.
\\\\Tuttavia, quando un contandino vorrà mettere a disposizione le sue patate nel contenitore, potrebbe ricevere in cambio meno mele rispetto alle patate inviate.
Questo perché il prezzo della mela potrebbe aumentare, e dunque per bilanciare il contenitore il prezzo delle patate dovrà di conseguenza diminuire.
\\Allo stesso modo, se un contandino volesse mettere a disposizione le proprie mele, riceverà più patate di quelle che avrebbe ricevuto inizialmente, considerato l'aumento del prezzo della mela rispetto alle patate.
\\\\Nella realtà dei fatti, questo contenitore magico è conosciuto come \textit{pool di liquidità}.

\section{Pool di liquidità}

\section{Protocollo Uniswap v3}

\subsection{Liquidità concentrata}

\subsection{Range Orders}

\subsection{Liquidity Minting}










\chapter{Orbit, piattaforma per automatizzare le strategie DeFi}

\chapter{Tecnologie utilizzate nello sviluppo di Orbit}

\chapter{Architettura degli Smart Contracts di Orbit}

\chapter{Future implementazione all'interno di Orbit}


\bibliographystyle{unsrtnat}
\bibliography{bibliografia}




\end{document}