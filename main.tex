\documentclass{report}


\usepackage[italian]{babel}

% Set page size and margins
% Replace `letterpaper' with `a4paper' for UK/EU standard size
\usepackage[letterpaper,top=2cm,bottom=2cm,left=3cm,right=3cm,marginparwidth=1.75cm]{geometry}

% Useful packages
\usepackage{amsmath}
\usepackage{graphicx}
\usepackage[colorlinks=true, allcolors=blue]{hyperref}
\usepackage[T1]{fontenc}
\usepackage[utf8]{inputenc}
\usepackage{setspace}
\usepackage[paper=a4paper,margin=1in]{geometry}
\usepackage[square,numbers,super]{natbib}
\usepackage[nottoc]{tocbibind}



\begin{document}


\begin{titlepage}
        
        \noindent
        \begin{minipage}[t]{0.19\textwidth}
            \vspace{-4mm}{\includegraphics[scale=1.15]{logo_unimib.pdf}}
        \end{minipage}
        \begin{minipage}[t]{0.81\textwidth}
        {
                \setstretch{1.42}
                {\textsc{Università degli Studi di Milano - Bicocca}} \\
                \textbf{Scuola di Scienze} \\
                \textbf{Dipartimento di Informatica, Sistemistica e Comunicazione} \\
                \textbf{Corso di laurea in Informatica} \\
                \par
        }
        \end{minipage}
        
	\vspace{40mm}
        
	\begin{center}
            {\LARGE{
                    \setstretch{1.2}
                    \textbf{Strategie di automatizzazione di liquidità nella finanza decentralizzata}
                    \par
            }}
        \end{center}
        
        \vspace{50mm}

        \noindent
        {\large \textbf{Relatore:} Prof. Alberto Leporati } \\
        
        \vspace{15mm}

        \begin{flushright}
            {\large \textbf{Relazione della prova finale di:}} \\
            \large{Christian Kobril} \\
            \large{Matricola 856448} 
        \end{flushright}
        
        \vspace{40mm}
        \begin{center}
            {\large{\bf Anno Accademico 2021-2022}}
        \end{center}

        \restoregeometry
        
    \end{titlepage}

\tableofcontents

\begin{enumerate}
\item Introduzione alla Finanza Decentralizzata
\item AMM e pool di liquidità
\item Complessità di Uniswap v3
\item Orbit, piattaforma per automatizzare e ottimizzare strategie defi
\item Utilizzo degli smart vaults
\item Moduli di Orbit: autocompound, rebalance and idle liquidity
\item Tecnologie utilizzate nello sviluppo di Orbit
\item Architettura degli smart contracts
\item Future implementazioni all'interno di Orbit
\end{enumerate}





\chapter{Introduzione}
Il tema centrale di questa tesi di laurea è la \textbf{finanza decentralizzata}.\\ In particolare, si approfondirà il ruolo che la blockchain ha in questo settore, fornendo esempi di un prodotto software concretamente sviluppato durante il mio  \textit{Project Work} svolto presso l'azienda \textit{Five Elements Labs Srl}.
\\\\Inoltre, verranno approfonditi i concetti di \textit{pool di liquidità}, \textit{automated market maker(AMM)} e \textit{strategie DeFi} su Uniswap v3.

\section{Cos'è la Finanza Decentralizzata}

Il termine \textbf{finanza decentralizzata} viene usato per classificare tutti quei servizi finanziari che avvengono direttamente tra due entità sulla blockchain.

\subsection{Differenze tra Finanza Decentralizzata e Tradizionale}

Per comprendere il concetto su cui si basa la finanza decentralizzata (da ora \textit{DeFi}), è bene dare un rapido sguardo alla sua controparte: la finanza tradizionale (o centralizzata).\\\\
Nella finanza centralizzata, ogni singola operazione finanziaria tra due persone (bonifici, prestiti, scambio di risorse, mutui) richiede l'interazione con soggetti di terze parti (tipicamente banche o altri enti).\\
Ciò incrementa le già prolisse tempistiche burocratiche, oltre ad aggiungere i costi dovuti al servizio fornito dagli enti che permettono l'operazione.
\\\\Invece, la finanza decentralizzata permette l'interazione tra due soggetti senza l'intermediazione di un sistema centralizzato, bensì mediante un applicativo software costruito sopra la tecnologia blockchain, rendendo le operazioni rapide, pubbliche e sicure.


\section{Caratteristiche e vantaggi della DeFi}

\subsection{Applicazioni decentralizzate}
Gli applicativi software utilizzati in DeFi vengono definitivi \textit{dApps (decentralized applications)}, ovvero particolari prodotti che utilizzano la blockchain di \textbf{Ethereum}\cite{ethereum}, una delle principali criptovalute, nota per la sua flessibilità e accessibilità.\\\\
Su tale blockchain risiedono dei particolari programmi denominati \textit{Smart Contracts}\cite{smartcontracts}, i quali si occupano di garantire sicurezza, trasparenza e irreversibilità delle operazioni avvenute sulla blockchain Ethereum.\\

\subsection{Accesso alle dApps}

Il concetto di login ideato nel web2, tipicamente caratterizzato dall'inserimento di un'email e una password, viene sorpassato da una nuova autenticazione del \textbf{web3}\cite{web2_web3}: attraverso il proprio \textit{portafoglio digitale} (da ora, wallet \cite{wallet}) è possibile accedere al proprio account e gestire i propri assets digitali, eventualmente mettendoli a disposizione della dApp a cui si è connessi.
\\\\I wallet presentano il vantaggio di non dover fornire nomi, indirizzi fisici, email o altre informazioni personali, garantendo la riservatezza dei propri dati; basta creare un wallet per avere immediato accesso alle piattaforme, senza registrazioni.



\subsection{Operazioni in DeFi}

Ogni operazione nella DeFi viene detta \textbf{transazione}. Una transazione è permanentemente salvata sui registri della blockchain, rendendo ogni singola operazione, associata ad un identificativo, consultabile in qualsiasi momento da qualsiasi persona.
\\Tale trasparenza viene difficilmente concessa dalle banche, ponendo la DeFi come un sistema aperto e rintracciabile.

\subsection{Flessibilità}

L'ultima caratteristica della DeFi che ritengo importante citare è ciò che più la contraddistingue dalla finanza tradizionale: la sua flessibilità.\\
In qualsiasi momento un utente può trasferire i propri assets digitali, senza dover chiedere il permesso a soggetti di terze parti, evitando costose commissioni e con un attesa che va dai pochi secondi ai pochi minuti. 


\chapter{Uniswap, piattaforma di Liquidity Providing}

Avendo approfondito cosa è la DeFi, quali sono le sue caratteristiche e i principali vantaggi, ritengo necessario concentrarsi su quali sono i prodotti "dApps" che hanno messo le basi per il lavoro svolto durante il mio Project Work.

\section{Uniswap}

Innanzitutto, è bene distinguere la piattaforma Uniswap\cite{uniswap} dall'omonimo protocollo.
\\La dApp di Uniswap (conosciuta come Uniswap Interface) è una piattaforma che permette agli utenti l'interazione con il protocollo Uniswap.
\\\\Quest'ultimo è bensì una suite di Smart Contracts, per definizione persistenti e non aggiornabili, che insieme costruiscono un \textbf{Automated Market Maker} (da ora AMM)\cite{amm}.

\section{Scambi nei mercati tradizionali}

La maggior parte dei mercati tradizionali ad accesso pubblico utilizza ciò che viene definito \textit{Order Book}\cite{order_book}, ossia un elenco degli ordini di acquisto e vendita attualmente aperti per un asset, organizzati per prezzo.
\\Sostanzialmente, un \textit{sistema di corrispondenza}\cite{matching} si occupa di abbinare gli ordini di acquisto con quelli di vendita, usando l'order book per eseguire le operazioni tra i partecipanti dello scambio.


\section{Automated Market Maker}

La rivoluzione introdotta dalla blockchain sta nella possibilità di creare nuovi tipi di scambi che abbinano algoritmicamente ordini di acquisto e vendita utilizzando gli smart contracts.
\\Tali scambi vengono detti \textit{Scambi Decentralizzati (DEX)}.
\\\\Un AMM è un protocollo DEX che si basa su un algoritmo di valutazione per prezzare gli asset mediante una formula matematica.
\\Il citato order book viene rimpiazzato con una pool di liquidità\cite{liquidity_pool}, contenente due asset, entrambi valutati l'uno rispetto all'altro.
\\Quando un asset viene scambiato per un altro, i prezzi relativi dei due asset cambiano, e viene determinato un nuovo tasso di mercato per entrambi.
In questo modo acquirenti e venditori interagiscono direttamente con la pool (e di conseguenza gli smart contracts), senza dover interagire tra di loro in modo diretto.

\subsection{Esempio pratico di AMM}

Un esempio pratico che mi ha aiutato a comprendere il funzionamento degli AMM è quello dei contadini di mele e patate.
\\Immaginiamo di essere un contadino e di avere solo patate, senza la possibilità di coltivare, e di conseguenza mangiare, nient'altro.\\ Un giorno ci viene proposto di effettuare degli scambi con un venditore di mele attraverso un messaggero, il quale decide di custodire mele e patate in un contenitore magico, in modo tale che rimangano a disposizione senza marcire (e dunque perdere di valore).
\\\\La regola fondamentale per questo scambio è una sola: \textit{il contenitore magico dovrà sempre contenere lo stesso valore di mele e patate.}
\\Tale regola è in realtà la formula alla base dell'AMM di Uniswap, conosciuta come \textbf{Constant Product Formula}: 

\[ x * y = k \]
\\dove x corrisponde al numero di mele e y al numero di patate nel contenitore.
\\Inizialmente il contenitore sarà perfettamente bilanciato, per esempio con 500 mele e 500 patate, entrambe prezzate ad €1 per un valore totale di €250.000.
\\\\Tuttavia, se un contadino volesse scambiare le sue patate grazie al contenitore, potrebbe ricevere in cambio meno mele rispetto alle patate inviate.
Questo perché il prezzo della mela potrebbe aumentare, e dunque per bilanciare il contenitore il prezzo delle patate dovrà di conseguenza diminuire.
\\Allo stesso modo, se un contadino volesse scambiare le proprie mele, riceverebbe più patate di quelle che avrebbe ricevuto inizialmente, considerato l'aumento del prezzo della mela rispetto alle patate.
\\\\Nella realtà dei fatti, questo contenitore magico è conosciuto come \textit{pool di liquidità}.

\section{Pool di liquidità}

È possibile vedere una pool di liquidità come uno spazio in cui i contandini (trader) possono mettere a disposizione la propria liquidità di mele e patate (cripto valute, nello specifico token ERC-20\cite{erc_20}); tali utenti vengono definiti fornitori di liquidità (\textit{liquidity providers, da ora LP}).
\\\\Come ricompensa per la liquidità fornita, gli \textbf{LP} ricevono commissioni sulle operazioni che avvengono nella pool a cui partecipano. 
Tali commissioni si applicano sulle singole transazioni effettuate con la liquidità fornita da un LP, e possono variare di percentuale dal 0.01\%\ fino all'1\%\.\\\\Nel caso di Uniswap, gli LP depositano un valore equivalente di due token; per esempio, 50\% ETH e 50\% USDC nella pool ETH/USDC.


\section{Protocollo Uniswap v3}

Uniswap v3 è l'ultima versione del protocollo rilasciata da Uniswap nel maggio 2021.\\ Tale protocollo definisce le funzionalità della suite di smart contracts con cui gli utenti interagiscono, introducendo importanti novità rispetto al suo predecessore, v2.

\subsection{Posizioni}

Utilizzando l'interfaccia Uniswap, gli utenti possono connettere il loro personale wallet per mettere a disposizione un certo ammontare di liquidità all'interno di una pool.
Tale liquidità, come spiegato in precedenza, dovrà mantenere un'equa proporzione tra i due asset messi a disposizione: tale operazione viene definita come \textit{apertura di una \textbf{posizione}} (o in inglese, position minting).
\\\\Su Uniswap v3, le posizioni vengono rappresentate mediante NFT (ERC-721\cite{erc_721}), i quali certificano un determinato wallet, in questo caso chi effettua il minting, come proprietario della posizione.

\subsection{Complicazioni di Uniswap v2}

\\\\In precedenza, nella v2 di Uniswap, i LP poteva mettere a disposizione i propri asset per scambi \textbf{a qualsiasi prezzo}.
In questo modo non vi era alcuna perdita di liquidità, portando però un importante svantaggio: la maggior parte della liquidità non veniva mai utilizzata negli scambi.
\\\\Provando a considerare una pool contenente una coppia di due stable coins, ossia token il quale prezzo rimane relativamente costante nel tempo, possiamo assumere che la liquidità all'esterno del tipico intervallo di prezzo dei suddetti stable coins non verrebbe mai toccata.
\\\\Per esempio in Uniswap v2 la coppia DAI/USDC utilizza circa il 0.50\%\ del capitale totale disponibile per gli scambi all'interno del range tra \$\00.99 e \$\11.01\cite{v2_waste}. Il resto della liquidità è distribuito nella restante fascia di prezzo tra 0 e $\infty$ (escluso il range sopra citato), rendendo quel capitale inutilizzabile (e dunque, non consentendo agli LP di guadagnare commissioni). 

\subsection{Liquidità concentrata}
\\\\Ciò che rende Uniswap v3 un protocollo davvero valido è l'idea della \textit{Liquidità Concentrata}\cite{concentrated_liquidity}. Tale liquidità viene distribuita in un intervallo di prezzo personalizzabile, a scelta dell'utente.
\\\\Riprendendo l'esempio sopra citato, un trader potrebbe decidere di investire nella pool DAI/USDC, scegliendo come range il più proficuo, ossia quello compreso tra \$\00.99 e \$\11.01.
In tal modo, la liquidità concentrata garantirà un guadagno superiori di commissioni (da ora fees) con il capitale da loro messo a disposizione.

\subsection{Tick di prezzo}

Per rendere la liquidità concentrata funzionale, lo spazio continuo del prezzo è stato partizionato in \textbf{tick}.
\\\\I tick sono i limiti di aree discrete nello spazio del prezzo. Tali limiti sono posizionati in modo tale che il diminuire o aumentare di 1 tick rappresenti l'aumento o la diminuzione percentuale del 0.01\%\ del prezzo in ogni punto dello spazio.
\\Dunque quando una posizione viene creata, un LP non può scegliere qualsiasi valore per il range di prezzo: è necessario che il limite inferiore (\textbf{lower tick}) e il limite superiore (\textbf{upper tick}) corrispondano a dei tick di prezzo validi.

\subsection{Swap e Fees}

Il modo più utilizzato per interagire con Uniswap v3 è tramite gli scambi (da ora \textbf{swap}).
Uno swap è relativamente semplice: un utente seleziona un token ERC-20 del quale è proprietario e un token che vorrebbe scambiare per esso. Uniswap venderà il token attualmente in possesso dell'utente, restituendo una quantità proporzionale del token desiderato, sottraendo una \textbf{swap fee}, ossia quella percentuale riconosciuta ai LPs per aver messo a disposizione la loro liquidità con la quale è avvenuto lo scambio.
\\\\Tuttavia, la transazione potrebbe richiedere alcuni minuti, a seconda della rete su cui avviene, rilevando un fenomeno conosciuto come \textbf{slippage}.
\\Lo slippage è l'alterazione di prezzo di un token che avviene mentre la transazione è in attesa di essere completata. Tale alterazione ha una soglia di tolleranza dell'\%\11: superata tale soglia l'operazione viene rifiutata e lo swap annullato, onde evitare grosse perdite per l'utente.

\section{Complicazioni di Uniswap v3}

Qualora il prezzo di un token dovesse muoversi verso una direzione (di discesa o di salita), il proprietario della posizione si ritroverebbe con un ammontare superiore di uno dei due token rispetto all'altro, in quanto il prezzo dell'uno sull'altro cambierà, fino a quando l'intera liquidità sarà relativa a solo uno dei due asset.
\\\\Per esempio, se in una pool ETH/USDC il prezzo di ETH dovesse diminuire, la percentuale di liquidità relativa a ETH aumenterebbe, per bilanciare il valore immesso all'interno della posizione stessa.
Allo stesso modo, se ETH dovesse aumentare di valore, la percentuale di USDC aumenterebbe a sua volta.
\\\\Con l'aumentare o il diminuire del prezzo di un asset nella pool, tale prezzo potrebbe uscire dall'intervallo che un LP ha impostato per una certa posizione. Nel momento in cui una posizione si dovesse trovare fuori dall'intervallo scelto (\textbf{Out Of Range position}) la liquidità diventerebbe inattiva (\textbf{idle liquidity}) e l'utente proprietario di tale liquidità non guadagnerebbe più fees.
\\Tuttavia, nel momento in cui il valore del primo token sul secondo dovesse tornare nell'intervallo di prezzo iniziale, il LP tornerebbe a guadagnare fee.
\\\\È proprio dal problema della liquidità inattiva che nasce \textbf{Orbit DeFi}, il prodotto sviluppato durante il mio project work.







\chapter{Orbit, piattaforma per automatizzare le strategie DeFi}

Ho avuto l'opportunità di svolgere il mio Project Work per Five Elements Labs, azienda specializzata nella produzione di software nel mondo blockchain; in particolare nei settori DeFi e NFT.
\\Durante tale esperienza, mi sono unito allo sviluppo del loro principale prodotto: \textbf{Orbit}\cite{orbit_website}.

\section{Perché Orbit}

We started Orbit based on two convictions. One is the emergence of low fees environments that allows for faster and better automation. The other is the presence of protocols such as Uniswap v3, which brings capital efficiency and operation complexity to another level. As protocols get more efficient, most of the time, they also become complicated for end-users.
Orbit first version allows users to automate operations and strategies on Uniswap v3 positions. Without paying any fee, it is possible to gain an additional yield from concentrated liquidity and replicate existing practices in place in so-called “vaults.”
The following versions will build more trustless automated strategies based on Uniswap and other protocols and chains.

\chapter{Tecnologie utilizzate nello sviluppo di Orbit}

\chapter{Architettura degli Smart Contracts di Orbit}

\chapter{Future implementazione all'interno di Orbit}


\bibliographystyle{unsrtnat}
\bibliography{bibliografia}




\end{document}